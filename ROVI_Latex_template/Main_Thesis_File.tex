%-----------------------------------------------------------------------------------------------------------
% Trajectory Planning Assignment Report
% Student: Mehmet Bahadur Sun
%-----------------------------------------------------------------------------------------------------------
\documentclass[12pt]{report} 
\usepackage[a4paper, top=25mm, bottom=25mm, left=25mm, right=25mm]{geometry}
\headheight=15pt
\usepackage[utf8]{inputenc}

% Bibliography
\usepackage[backend=biber,style=nature]{biblatex}
\addbibresource{references.bib}

% Packages
\usepackage{fancyhdr}
\pagestyle{fancy}
\fancyhf{}
\usepackage{float}
\usepackage{subcaption}
\usepackage{color}
\usepackage{setspace}
\setstretch{1.15}
\usepackage{import}
\usepackage{titlesec}
\usepackage[export]{adjustbox}
\usepackage{amsmath}
\usepackage{amssymb}
\usepackage{graphicx}
\graphicspath{{./figures/}}
\usepackage{caption}
\usepackage[labelfont=bf]{caption}
\usepackage{algorithm}
\usepackage{algpseudocode}
\usepackage{listings}
\usepackage{xcolor}
\usepackage{hyperref}

% Code listing style
\lstset{
    basicstyle=\ttfamily\small,
    keywordstyle=\color{blue},
    commentstyle=\color{green!50!black},
    stringstyle=\color{red},
    breaklines=true,
    frame=single,
    numbers=left,
    numberstyle=\tiny\color{gray}
}

% Header/Footer
\fancyhead[R]{mehmetbahadursun@sdu.dk}
\fancyhead[L]{Assignment 1: Trajectory Planning}
\fancyfoot[C]{\thepage}
\renewcommand{\headrulewidth}{1pt}
\renewcommand{\footrulewidth}{0.5pt}

%-----------------------------------------------------------------------------------------------------------
\begin{document}

% Title Page
\begin{titlepage}
\begin{center}
    \vspace*{1cm}
    \LARGE\textbf{Assignment 1: Trajectory Generation for Pick and Place Tasks}
    
    \vspace{1cm}
    \includegraphics[width=0.6\textwidth]{Image/SDU_BLACK_RGB_png.png}
    
    \vspace{2cm}
    \large
    \textbf{Mehmet Bahadur Sun}\\
    \vspace{0.3cm}
    Student Number: XXXXXX\\
    mehmetbahadursun@sdu.dk
    
    \vfill
    \begin{flushleft}
    Robotics and Computer Vision 2024\\
    University of Southern Denmark (SDU)\\
    Odense\\
    Date of submission: XX.12.2024
    \end{flushleft}
\end{center}
\end{titlepage}

%-----------------------------------------------------------------------------------------------------------
% INTRODUCTION
%-----------------------------------------------------------------------------------------------------------
\section{Introduction}

Bu ödevde, bir UR5e robot kolu için iki farklı yörünge planlama yöntemi uygulanmıştır:
(1) \textbf{Trapez Hız Profili} ile nokta-nokta (P2P) interpolasyon ve 
(2) \textbf{RRT (Rapidly-exploring Random Tree)} algoritması ile engelden kaçınmalı yol planlama.

Amaç, verilen workcell ortamında üç farklı nesneyi (kutu, silindir, T-blok) alıp belirlenen bölgeye bırakmaktır. Her iki yöntem de tüm nesneler için test edilmiş ve performansları karşılaştırılmıştır.

% English version:
% This assignment implements two trajectory planning methods for a UR5e robot arm:
% (1) Point-to-Point interpolation with Trapezoidal Velocity Profile, and
% (2) RRT (Rapidly-exploring Random Tree) algorithm for collision-free path planning.

%-----------------------------------------------------------------------------------------------------------
% METHODS
%-----------------------------------------------------------------------------------------------------------
\section{Methods}

\subsection{Trapezoidal Velocity Profile}

\subsubsection{Motivation: Why Not Linear Interpolation?}

With linear interpolation, velocity is constant:
\begin{equation}
\dot{q} = \frac{q_f - q_i}{t_f - t_i} = \text{constant}
\end{equation}

However, at $t=0$, velocity jumps instantly from 0 to this value. Since this happens in zero time:
\begin{equation}
\ddot{q} = \frac{\Delta v}{\Delta t} = \frac{V}{0} = \infty \quad \text{(Infinite acceleration!)}
\end{equation}

This causes mechanical stress and motor damage. The trapezoidal profile solves this by limiting acceleration to a finite value.

\subsubsection{Alternative Methods}

\begin{table}[H]
\centering
\begin{tabular}{|l|l|l|}
\hline
\textbf{Method} & \textbf{Pros} & \textbf{Cons} \\
\hline
Linear & Simplest & Infinite acceleration \\
Trapezoidal & Simple, efficient & Discontinuous acceleration \\
Parabolic Blend & Same as trapezoidal & Position-based view \\
Cubic Polynomial & Smooth velocity & No via-point support \\
Quintic Polynomial & Smooth vel \& acc & High computation \\
S-Curve & Jerk-limited & Complex implementation \\
\hline
\end{tabular}
\caption{Comparison of trajectory generation methods}
\end{table}

\textit{Note: Parabolic Blend $\approx$ Trapezoidal Profile (different viewpoints of same concept)}

\subsubsection{Three Phases of Trapezoidal Profile}

\begin{enumerate}
    \item \textbf{Acceleration phase} ($0 \leq t \leq t_c$): Velocity increases linearly
    \item \textbf{Constant velocity phase} ($t_c < t \leq t_f - t_c$): Velocity stays constant
    \item \textbf{Deceleration phase} ($t_f - t_c < t \leq t_f$): Velocity decreases linearly
\end{enumerate}

\subsubsection{Matematiksel Formülasyon}

Pozisyon denklemi (Sciavicco, Eq. 4.8):
\begin{equation}
q(t) = \begin{cases}
q_i + \frac{1}{2}\ddot{q}_c t^2 & 0 \leq t \leq t_c \\
q_i + \ddot{q}_c t_c \left(t - \frac{t_c}{2}\right) & t_c < t \leq t_f - t_c \\
q_f - \frac{1}{2}\ddot{q}_c (t_f - t)^2 & t_f - t_c < t \leq t_f
\end{cases}
\end{equation}

Hız denklemi:
\begin{equation}
\dot{q}(t) = \begin{cases}
\ddot{q}_c t & 0 \leq t \leq t_c \\
\ddot{q}_c t_c = \dot{q}_c & t_c < t \leq t_f - t_c \\
\ddot{q}_c (t_f - t) & t_f - t_c < t \leq t_f
\end{cases}
\end{equation}

İvme denklemi:
\begin{equation}
\ddot{q}(t) = \begin{cases}
+\ddot{q}_c & 0 \leq t \leq t_c \\
0 & t_c < t \leq t_f - t_c \\
-\ddot{q}_c & t_f - t_c < t \leq t_f
\end{cases}
\end{equation}

Hızlanma süresi hesabı (Eq. 4.6):
\begin{equation}
t_c = \frac{t_f}{2} - \frac{1}{2}\sqrt{\frac{t_f^2 \ddot{q}_c - 4(q_f - q_i)}{\ddot{q}_c}}
\end{equation}

\subsection{RRT Algorithm}

RRT, konfigürasyon uzayında rastgele bir ağaç büyüterek yol bulan olasılıksal bir algoritmadır.

\begin{algorithm}[H]
\caption{RRT Algorithm}
\begin{algorithmic}[1]
\State $T \gets \{q_{start}\}$
\While{$q_{goal} \notin T$}
    \State $q_{rand} \gets \text{RANDOM\_SAMPLE}()$
    \State $q_{near} \gets \text{NEAREST}(T, q_{rand})$
    \State $q_{new} \gets \text{EXTEND}(q_{near}, q_{rand})$
    \If{$\text{IS\_VALID}(q_{new})$}
        \State $T \gets T \cup \{q_{new}\}$
    \EndIf
\EndWhile
\State \Return path from $q_{start}$ to $q_{goal}$
\end{algorithmic}
\end{algorithm}

\subsection{Comparison of Methods}

\begin{table}[H]
\centering
\begin{tabular}{|l|c|c|}
\hline
\textbf{Özellik} & \textbf{P2P Trapez} & \textbf{RRT} \\
\hline
Tip & Deterministik & Olasılıksal \\
Engel Kaçınma & Hayır & Evet \\
Hesaplama Hızı & Çok Hızlı & Değişken \\
Yol Kalitesi & Pürüzsüz & Post-process gerekir \\
\hline
\end{tabular}
\caption{İki yöntemin karşılaştırması}
\end{table}

%-----------------------------------------------------------------------------------------------------------
% RESULTS
%-----------------------------------------------------------------------------------------------------------
\section{Results and Discussion}

\subsection{Trapezoidal Profile Results}

% TODO: Add figure when generated
% \begin{figure}[H]
%     \centering
%     \includegraphics[width=0.9\textwidth]{figures/trapezoidal_profile.png}
%     \caption{Trapez hız profili: Pozisyon, hız ve ivme grafikleri}
% \end{figure}

\subsection{RRT Path Planning Results}

% TODO: Add figure when generated
% \begin{figure}[H]
%     \centering
%     \includegraphics[width=0.9\textwidth]{figures/rrt_path.png}
%     \caption{RRT ile bulunan yol}
% \end{figure}

\subsection{Comparison Plots}

% TODO: Add multi-joint trajectory comparison
% \begin{figure}[H]
%     \centering
%     \includegraphics[width=0.9\textwidth]{figures/comparison.png}
%     \caption{Her iki yöntemin eklem bazlı karşılaştırması}
% \end{figure}

%-----------------------------------------------------------------------------------------------------------
% CONCLUSION
%-----------------------------------------------------------------------------------------------------------
\section{Conclusion}

Bu çalışmada iki farklı yörünge planlama yöntemi uygulanmış ve karşılaştırılmıştır:

\begin{itemize}
    \item \textbf{Trapez hız profili}, önceden tanımlanmış waypoint'ler arasında pürüzsüz ve öngörülebilir hareket sağlar.
    \item \textbf{RRT}, engelli ortamlarda çarpışmasız yollar bulabilir ancak bulunan yollar genellikle düzensizdir.
\end{itemize}

Gelecek iyileştirmeler:
\begin{itemize}
    \item RRT yollarına post-processing (smoothing) eklenmesi
    \item Gerçek zamanlı replanning desteği
    \item Dinamik engel takibi
\end{itemize}

%-----------------------------------------------------------------------------------------------------------
\newpage
\printbibliography[title={References}]

\end{document}